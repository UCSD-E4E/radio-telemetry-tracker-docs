%
%  @file Python_Style_Guide.tex
%
%  @author Nathan Hui, nthui@eng.ucsd.edu
% 
%  @description 
%  This is the Python Style guide for the Radio Telemetry Project.
%
%  This program is free software: you can redistribute it and/or modify
%  it under the terms of the GNU General Public License as published by
%  the Free Software Foundation, either version 3 of the License, or
%  (at your option) any later version.
%
%  This program is distributed in the hope that it will be useful,
%  but WITHOUT ANY WARRANTY; without even the implied warranty of
%  MERCHANTABILITY or FITNESS FOR A PARTICULAR PURPOSE.  See the
%  GNU General Public License for more details.
%
%  You should have received a copy of the GNU General Public License
%  along with this program.  If not, see <http://www.gnu.org/licenses/>.
%
%
%  DATE      WHO DESCRIPTION
%  ----------------------------------------------------------------------------
%  06/10/20  NH  Initial commit
%

\documentclass{article}
\title{Radio Telemetry Tracker C/C++ Style Guide}
\author{Nathan Hui, nthui@eng.ucsd.edu}
\date{\today\\v1.0a}
\usepackage{listings,fullpage,xcolor}
% Default fixed font does not support bold face


% Custom colors
\usepackage{color}
\definecolor{deepblue}{rgb}{0,0,0.5}
\definecolor{deepred}{rgb}{0.6,0,0}
\definecolor{deepgreen}{rgb}{0,0.5,0}
\lstdefinestyle{custompython}{
    language=Python,
    basicstyle=\footnotesize\ttfamily,
    otherkeywords={self},             % Add keywords here
    keywordstyle=\bfseries\color{deepblue},
    emph={MyClass,__init__},          % Custom highlighting
    emphstyle=\bfseries\color{deepred},    % Custom highlighting style
    stringstyle=\color{deepgreen},
    frame=L,                         % Any extra options here
    showstringspaces=false            % 
    commentstyle=\itshape\color{purple!40!black},
    numbers=left
}
\lstset{
    style=custompython
}
\begin{document}
\maketitle
\section{Philosopy}
In general, code should be written in a way that clearly expresses the local intent of the code, and allows for easy readability.  The following are rules that should always be followed when writing code for the RTT project.  Cases not covered by the below rules should be written in a way that is easy to understand and clearly expresses the intent behind the code.

The single most important rule when writing code is this: check the surrounding code and try to imitate it.
\section{Code Formatting}
\section{Code Namespace}
\section{Documentation}
\begin{enumerate}
    \item Functions should be documented both at definition and at declaration.
    \item Use comments to document the code throughout functions, particularly where the code is not immediately obvious.
    \item The top of every file should have the following comment block:
\begin{lstlisting}
###############################################################################
#     Radio Collar Tracker [component]
#     Copyright (C) [year]  [author], [email]
#
#     This program is free software: you can redistribute it and/or modify
#     it under the terms of the GNU General Public License as published by
#     the Free Software Foundation, either version 3 of the License, or
#     (at your option) any later version.
#
#     This program is distributed in the hope that it will be useful,
#     but WITHOUT ANY WARRANTY; without even the implied warranty of
#     MERCHANTABILITY or FITNESS FOR A PARTICULAR PURPOSE.  See the
#     GNU General Public License for more details.
#
#     You should have received a copy of the GNU General Public License
#     along with this program.  If not, see <http://www.gnu.org/licenses/>.
#
###############################################################################
#
# DATE      WHO Description
# -----------------------------------------------------------------------------
#
###############################################################################
\end{lstlisting}

    Every commit must be documented in the changelog at the top of the file.  The author and name fields should be populated with the original author of the code.  Subsequent edits should be denoted with initials in the changelog.

\end{enumerate}
\end{document}